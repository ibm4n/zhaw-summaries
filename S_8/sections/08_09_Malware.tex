 \section{Malware (Part I)}

\subsection{Overview and Goals}
{
\textbf{Definition:} Malware (malicious software) is code that compromises CIA (confidentiality, integrity, availability) or behaves without admin/user consent.

\textbf{Goals:}
\begin{itemize}[noitemsep]
  \item Understand common malware types: worms, Trojans, ransomware, rootkits, bootkits
  \item Understand malware communication strategies
  \item Understand why malware defense is hard
\end{itemize}
}

\subsection{Malware History (Milestones)}
{
\begin{itemize}[noitemsep]
  \item \textbf{1949 – Von Neumann:} Self-replicating programs (theoretical)
  \item \textbf{1982 – Elk Cloner:} First virus in the wild (Apple II, boot sector)
  \item \textbf{1988 – Morris Worm:} First internet worm, infected 2000 Unix systems
  \item \textbf{2001 – Win32.S-0-1:} First social network worm via MSN
\end{itemize}
}

\subsection{Malware Classification}
{
\textbf{Types of Classification:}
\begin{itemize}[noitemsep]
  \item \textbf{By Type:} virus, worm, Trojan, bot, etc.
  \item \textbf{By Behavior:} e.g., info stealer, downloader
  \item \textbf{By Family/Lineage:} code origin or evolution
\end{itemize}

\textbf{Note:} Categories are not mutually exclusive.
}

\subsection{Key Malware Types}
{
\begin{itemize}[noitemsep]
  \item \textbf{Trojan Horse:} Disguised as legitimate software
  \item \textbf{Backdoor/RAT:} Allows attacker remote control
  \item \textbf{Downloader:} Downloads more malicious tools
  \item \textbf{Dropper:} Installs malware locally from embedded data
  \item \textbf{Bot/Botnet:} Controlled fleet for DDoS, spam, credential theft
  \item \textbf{Spyware/Monitor:} Logs keystrokes, screen, audio, etc.
  \item \textbf{Information Stealer:} Auto-extracts specific data (e.g., cookies, documents)
  \item \textbf{Scareware/Adware:} Manipulates user with fake alerts or annoying ads
  \item \textbf{Ransomware:} Encrypts files and demands ransom (often using public-key crypto)
  \item \textbf{Virus:} Infects files and propagates with user assistance
  \item \textbf{Worm:} Self-replicating, spreads autonomously via vulnerabilities
\end{itemize}
}

\subsection{Advanced Malware Concepts}
{
\textbf{Living Off the Land:} Abuses legitimate tools (e.g., PowerShell)

\textbf{Fileless Malware:}
\begin{itemize}[noitemsep]
  \item Only resides in memory
  \item Injected via exploits or via legitimate software
\end{itemize}

\textbf{Cryptominer:} Uses resources to mine cryptocurrency

\textbf{Spambot/Mailer:} Sends email from compromised accounts
}

\subsection{Rootkits}
{
\textbf{Goal:} Stealth and persistence

\textbf{Types:}
\begin{itemize}[noitemsep]
  \item \textbf{User-Mode:} API hooking, runs with user privileges
  \item \textbf{Kernel-Mode:} SSDT hooking, device drivers
  \item \textbf{Bootkits:} Infect bootloader, early execution
  \item \textbf{Hypervisor (Ring -1):} Hides OS in VM (e.g., Blue Pill)
  \item \textbf{Firmware Rootkits:} BIOS, NIC, HDD, routers
\end{itemize}

\textbf{Detection:}
\begin{itemize}[noitemsep]
  \item Look for altered data structures (e.g., SSDT)
  \item Timing analysis
  \item Use of external time sources for hypervisor detection
\end{itemize}
}

\subsection{Malware Communication}
{
\textbf{Goals:}
\begin{itemize}[noitemsep]
  \item Ensure resilience to take-downs
  \item Remain undetected
\end{itemize}

\textbf{Architectures:}
\begin{itemize}[noitemsep]
  \item \textbf{Client-Server:} Direct communication with C2
  \item \textbf{Peer-to-Peer (P2P):} Resilient, harder to disrupt
\end{itemize}

\textbf{Evasion Techniques:}
\begin{itemize}[noitemsep]
  \item Fast Flux – rotating IPs rapidly via DNS
  \item DGA – generate new domains dynamically
  \item Domain Fronting – mask C2 as legitimate service
  \item Use of legit apps (Dropbox, Evernote, IRC)
\end{itemize}
}

\subsection{Covert Channels}
{
\textbf{Smart Communication:}
\begin{itemize}[noitemsep]
  \item Mimics “normal” network behavior
  \item Protocols: HTTP(S), DNS, SSH, etc.
\end{itemize}

\textbf{Covert Channels:}
\begin{itemize}[noitemsep]
  \item Delay-based exfiltration (e.g., WLAN inter-packet delays)
  \item DNS Covert Channels (data in DNS queries)
\end{itemize}

\textbf{Example:}
\begin{itemize}[noitemsep]
  \item \texttt{cl1020-getcmd-lastwasok.adversary.com} encodes commands
  \item Response can be IP-encoded instructions (e.g., \texttt{100.105.114.32})
\end{itemize}
}

\section{Malware Part 2}

\subsection{Overview}
{
\textbf{Goals:}
\begin{itemize}[noitemsep]
  \item Understand why malware defenses are still weak
  \item Learn how Anti-Virus (AV) works (signatures, fuzzy hashes, behavior, etc.)
  \item Recognize evasion techniques and AV limitations
\end{itemize}
}

\subsection{Detection Techniques}
{
\textbf{AV Systems Use:}
\begin{itemize}[noitemsep]
  \item \textbf{Static Analysis:} Without execution (file metadata, binary/code)
  \item \textbf{Dynamic Analysis:} With execution (memory, syscalls, network)
\end{itemize}

\textbf{Detection Engines:}
\begin{itemize}[noitemsep]
  \item \textbf{Signature-based:} Exact/fuzzy match to known byte sequences
  \item \textbf{Heuristic-based:} Rules from domain experts (structure, imports)
  \item \textbf{Behavior-based:} Detects what malware \emph{does}
  \item \textbf{Reputation-based:} Based on file origin, age, prevalence
\end{itemize}
}

\subsection{Anti-Virus Architecture}
{
\begin{itemize}[noitemsep]
  \item Host + Network based components
  \item \textbf{Cloud AV:} Submits file metadata (fuzzy hashes, origin, behavior)
  \item Unknown files quarantined and uploaded
  \item Signature updates allow instant response post-``patient zero''
\end{itemize}
}

\subsection{Signatures}
{
\textbf{Traditional:}
\begin{itemize}[noitemsep]
  \item Byte sequences, hash matches (e.g., MD5, SHA-1)
\end{itemize}

\textbf{Fuzzy Hashes (CTPH):}
\begin{itemize}[noitemsep]
  \item \textbf{ssdeep:} Compares pieces using rolling hash and edit distance
  \item \textbf{sdhash:} Bloom filters from rare byte sequences
  \item \textbf{TLSH:} N-gram frequency distribution
\end{itemize}

\textbf{YARA Rules:}
\begin{itemize}[noitemsep]
  \item Rule-based matching (conditions, strings)
  \item Used in malware classification, Office analysis, pcap, etc.
\end{itemize}
}

\subsection{Heuristic and Behavioral Detection}
{
\textbf{Heuristic:}
\begin{itemize}[noitemsep]
  \item Static: File structure and metadata anomalies
  \item Dynamic: Simulated execution to observe rules
\end{itemize}

\textbf{Behavioral:}
\begin{itemize}[noitemsep]
  \item Observes runtime actions: file/registry access, networking
  \item Can be performed in sandbox (e.g., Cuckoo)
\end{itemize}
}

\subsection{Reputation and ML}
{
\textbf{Reputation-based Detection:}
\begin{itemize}[noitemsep]
  \item Based on age, prevalence, and origin
\end{itemize}

\textbf{Machine Learning:}
\begin{itemize}[noitemsep]
  \item Static + dynamic features used to train classifiers
  \item Can learn new variants without manual rule updates
\end{itemize}
}

\subsection{Evasion Techniques}
{
\begin{itemize}[noitemsep]
  \item \textbf{File Format Tricks:} Rename, embed in obscure types
  \item \textbf{Compression:} Zip bombs, password-protected archives
  \item \textbf{Polymorphism:} Self-mutating payload
  \item \textbf{Metamorphism:} Full code mutation (not just payload)
  \item \textbf{Sandbox Detection:} Check for mouse/keyboard input, clock, registry, VMs
  \item \textbf{Timing Tricks:} Sleep until analysis period is over
\end{itemize}
}

\subsection{Effectiveness of AV}
{
\begin{itemize}[noitemsep]
  \item AV is effective but imperfect
  \item No tool guarantees full protection
  \item Test results vary by setup, are often vendor-sponsored
  \item Retrospective testing hard due to update mechanisms
\end{itemize}
}
