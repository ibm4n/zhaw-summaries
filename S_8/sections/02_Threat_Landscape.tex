\section{Threat Landscape}
\subsection{Definition}
{
\textbf{Definition:} Collection of threats in a domain/context
\begin{itemize}[noitemsep]
  \item Focus: Threat types, agents, vectors (not mitigations)
  \item Supports risk evaluation:
  \begin{itemize}[noitemsep]
    \item Risk = Threat $\times$ Vulnerability $\times$ Consequence
    \item Risk = Likelihood $\times$ Impact
  \end{itemize}
\end{itemize}
}

\subsection{Threat Agents}
{
\textbf{Attributes:} Motivation, Resources, Skill, Role

\begin{itemize}[noitemsep]
  \item \textbf{Cyber Criminals:}  money,secrets, medium-high skill/resources, *-as-a-Service
  \item \textbf{Online Social Hackers:} High social, low-medium tech skill, psychology-based attacks
  \item \textbf{Cyber Spies:} State/corp, espionage, very high skill/resources
  \item \textbf{Employees:} Insider threat, low-medium skill, intentional/unintentional
  \item \textbf{Script Kiddies:} Low skill, use public tools, motive: fun/fame
  \item \textbf{Others:} 
  \begin{itemize}[noitemsep]
    \item Hacktivists - political/societal goals
    \item Cyber Fighters - nationalists (non-state)
    \item Cyber Terrorists - fear/political damage
  \end{itemize}
\end{itemize}
}

\subsection{Cyber Kill Chain}
{
\textbf{7 Steps of an Attack:}
\begin{enumerate}[noitemsep]
  \item Reconnaissance - gather info
  \item Weaponization - create exploit + payload
  \item Delivery - transmit payload (email, USB...)
  \item Exploitation - trigger vuln.
  \item Installation - install malware
  \item Command \& Control - remote channel
  \item Actions on Objectives - data theft, damage
\end{enumerate}

\textbf{Defenders can break the chain at any step.}
}
