\section{Mobile Platform Security}



\subsection{Mobile Device Security Characteristics}
{
\begin{itemize}[noitemsep]
  \item Similar hardware/software to standard computers but used differently
  \item Frequently lost or stolen, contain personal/sensitive data
  \item Used across various networks including untrusted Wi-Fi
  \item Store credentials, easy app installation, low user awareness
  \item High-value attack targets
  \item Vendors implement protection beyond standard computers:
  \begin{itemize}[noitemsep]
    \item Prevent malicious app installation
    \item Prevent data access if device is stolen
    \item Restrict admin rights for users
  \end{itemize}
  \end{itemize}
}


\subsection{iOS Security Model}
{
\textbf{Principles:}
\begin{itemize}[noitemsep]
  \item Secure Boot Process
  \item File Encryption and Data Protection
  \item Passcode, Face ID / Touch ID
  \item Keychain
  \item Signed Apps only (App Store)
  \item Sandboxing and Permissions
\end{itemize}

\textbf{Secure Boot Chain:}
\begin{itemize}[noitemsep]
  \item Boot ROM (immutable, contains Apple Root CA key) $\rightarrow$  iBoot $\rightarrow$  Kernel
  \item Ensures only trusted iOS kernel is executed
  \item Vulnerabilities in Boot ROM can lead to jailbreaks (e.g., Checkm8)
\end{itemize}

\textbf{File Encryption:}
\begin{itemize}[noitemsep]
  \item AES-256 hardware encryption between memory and flash
  \item Unique UID (per device) and GID (per processor class)
  \item Encryption uses per-file keys wrapped with class keys, further encrypted with UID/passcode
  \item Metadata encrypted with a file system key (stored in effaceable storage)
  \item Secure Enclave manages sensitive key operations
\end{itemize}

\textbf{Passcodes and Protection Classes:}
\begin{itemize}[noitemsep]
  \item 4/6-digit or alphanumeric codes used as extra key material
  \item Protection classes: Complete, Until First Unlock, No Protection
  \item Secure Enclave enforces delays to slow down brute-force attacks
\end{itemize}

\textbf{Keychain:}
\begin{itemize}[noitemsep]
  \item Secure password/credentials storage
  \item Tied to app ID and protection classes
  \item Variants: Device-only, Passcode-protected, Encrypted backups
\end{itemize}

\textbf{App Code Signing and App Store:}
\begin{itemize}[noitemsep]
  \item Only Apple-signed or Apple-certified apps run
  \item App Store reviews apps for private APIs or suspicious behavior
\end{itemize}

\textbf{Runtime Security:}
\begin{itemize}[noitemsep]
  \item Mandatory Access Control (TrustedBSD MAC)
  \item Sandboxing enforced per-app via kernel policies
  \item ASLR, NX-bit, WX memory, signed code enforcement
\end{itemize}

\textbf{Inter-App Communication:}
\begin{itemize}[noitemsep]
  \item Limited to URL schemes
  \item Receiving app must register scheme, can validate sender
\end{itemize}

\textbf{Jailbreaking:}
\begin{itemize}[noitemsep]
  \item Escalate privileges via vulnerabilities (BootROM, iBoot, userland)
  \item Allows root rights, unsigned code, disables sandboxing
  \item Tools: Cydia, OpenSSH
  \item Decreases security; not recommended for production devices
\end{itemize}
}

\subsection{Android Security Model}
{
\textbf{Principles:}
\begin{itemize}[noitemsep]
  \item Runtime security features
  \item Permissions model
  \item App code signing
  \item Limited inter-app communication
  \item Full Disk and File-Based Encryption
\end{itemize}

\textbf{Runtime and Sandboxing:}
\begin{itemize}[noitemsep]
  \item Based on modified Linux kernel
  \item Apps run in ART VM, each app has a dedicated Linux user
  \item Sandboxing enforced via Linux DAC
  \item Uses APKs/AABs with dedicated data directories
\end{itemize}

\textbf{Security Features:}
\begin{itemize}[noitemsep]
  \item ASLR, NX-bit for native code
  \item Java code safer, but native C/C++ code still used
\end{itemize}

\textbf{Permissions:}
\begin{itemize}[noitemsep]
  \item Pre-6.0: Accept all at install
  \item Post-6.0: Runtime permissions; dangerous vs. normal
  \item Dangerous permissions grouped; one grants all in group
\end{itemize}

\textbf{Code Signing and Distribution:}
\begin{itemize}[noitemsep]
  \item Self-signed certificates (not by Google)
  \item Signing ensures app updates only by same developer
  \item Apps can share user IDs if signed with same key
  \item Apps can be installed from anywhere (Play Store, side-load)
\end{itemize}

\textbf{App Store Security:}
\begin{itemize}[noitemsep]
  \item Play Store reviews apps but not foolproof
  \item Repackaged APKs, hidden behavior, runtime code loading
  \item Malware can be injected post-approval
\end{itemize}

\textbf{Inter-App Communication:}
\begin{itemize}[noitemsep]
  \item Uses Intents for messaging
  \item Implicit (via URL schemes) and explicit (targeting component)
  \item Allows bidirectional communication
\end{itemize}

\textbf{File Encryption:}
\begin{itemize}[noitemsep]
  \item Android 5–9: Full Disk Encryption (FDE) via dm-crypt
  \item Android 10+: File-Based Encryption (FBE) using fscrypt
  \item FDE key stretch via scrypt; stored in TEE if available
  \item Weak passwords = weak protection
\end{itemize}

\textbf{Rooting:}
\begin{itemize}[noitemsep]
  \item Grants user root access, disables restrictions
  \item Done via bootloader unlock or privilege escalation
  \item Often uses \texttt{su} binary + SuperSU manager
\end{itemize}
}

\subsection{Summary}
{
\begin{itemize}[noitemsep]
  \item Mobile devices face higher risks than desktops (loss, theft, sensitive data)
  \item iOS: Strong, restrictive security model
  \item Android: Flexible, but weaker model due to diversity
  \item Jailbreaking and rooting remove built-in protections
\end{itemize}
}
