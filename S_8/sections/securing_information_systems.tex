\section{SECURING INFORMATION SYSTEMS}

\subsection{Information System}
{
\textbf{Definition:} Structured set of components to collect, process, store, communicate information
\begin{itemize}[noitemsep]
  \item Applications, services, IT assets
  \item Software, hardware
  \item Data, methods, procedures
  \item People (users, operators)
\end{itemize}
}
\subsection{Information Security Management System (ISMS)}
{
\textbf{Definition:} Structured approach to manage information security
\begin{itemize}[noitemsep]
  \item Risk management framework
  \item Includes: people, processes, technology
  \item Goal: keep risks at acceptable levels
  \item Implemented by management (typically CISO)
  \item Checklist-style, high abstraction
  \item Not a technical solution
\end{itemize}
}
\subsection{Security Controls}
{
\textbf{Definition:} Countermeasures to reduce, detect, respond to risks
\begin{itemize}[noitemsep]
  \item \textbf{Types:}
    \begin{itemize}[noitemsep]
      \item Preventive - stop incidents (e.g., firewalls, auth)
      \item Detective - identify incidents (e.g., IDS)
      \item Corrective - limit damage (e.g., backups)
    \end{itemize}
  \item \textbf{Attributes:}
    \begin{itemize}[noitemsep]
      \item Security Property: CIA
      \item Function: Identify, Protect, Detect, Respond, Recover
      \item Category: People, Physical, Technology, Organizational
    \end{itemize}
\end{itemize}
}
\subsection{ISO 27000 Series}
{
\textbf{ISO 27001:} Lists high-level controls (e.g., disposal, network security) \\
\textbf{ISO 27002:} Implementation guidance for ISO 27001 controls
\begin{itemize}[noitemsep]
  \item Abstract, generic - industry-independent
  \item Checklist-like reference
  \item Example: Malware protection - anti-virus, user training
\end{itemize}
}
\subsection{CIS Controls}
{
Best-practice guidelines whose development started in 2008

\textbf{Definition:} Practical, prioritized controls from real-world attacks
\begin{itemize}[noitemsep]
  \item \textbf{Groups:}
    \begin{itemize}[noitemsep]
      \item IG1 - Basic hygiene (SMEs)
      \item IG2 - Mid-level, enterprise-grade
      \item IG3 - Advanced protection, targeted threats
    \end{itemize}
  \item \textbf{Examples:}
    \begin{itemize}[noitemsep]
      \item CSC 1 - Inventory of devices (active + passive)
      \item CSC 2 - Inventory of software (whitelisting)
      \item CSC 7 - Continuous vuln. management (scanners, patching)
    \end{itemize}
\end{itemize}
}
\subsection{Measuring Security}
{
\textbf{Challenge:} Measuring security = hard / approximate
\begin{itemize}[noitemsep]
  \item \textbf{Methods:}
    \begin{itemize}[noitemsep]
      \item Audits (compliance vs. standards)
      \item Penetration testing
      \item Risk = Likelihood $\times$ Impact
    \end{itemize}
  \item \textbf{Metrics:}
    \begin{itemize}[noitemsep]
      \item \% vulnerabilities patched in time (NIST SP 800-55)
      \item Ratio blocked/successful malware (ISO 27004)
    \end{itemize}
  \item \textbf{Purpose:}
    \begin{itemize}[noitemsep]
      \item Assess control effectiveness
      \item Demonstrate compliance
      \item Guide security decisions
    \end{itemize}
\end{itemize}
}
\subsection{Key Takeaways}
{
\begin{itemize}[noitemsep]
  \item Securing systems = people + process + tech
  \item ISMS / CIS = frameworks, not full solutions
  \item Controls must be context-specific + prioritized
  \item Measuring helps track + improve security posture
\end{itemize}
}


