\section{Security Controls \& SIEM}

\subsection{Fundamental Control Principles}
{
\begin{itemize}[noitemsep]
  \item \textbf{Least Privilege} – minimum necessary access
  \item \textbf{Fail-Safe Defaults} – deny by default
  \item \textbf{Complete Mediation} – every access checked
  \item \textbf{Separation of Privilege} – multiple conditions for access
  \item \textbf{Least Common Mechanism} – minimize shared components
  \item \textbf{Open Design} – transparency over obscurity
  \item \textbf{Psychological Acceptability} – usability of security
  \item \textbf{Goal:} reduce attack surface, enforce secure defaults
\end{itemize}
}

\subsection{SIEM Overview}
{
\textbf{Definition:} \textbf{SIEM} = Security Information \& Event Management

\begin{itemize}[noitemsep]
  \item Collects, normalizes, stores, correlates, and analyzes security data
  \item Central component of SOC (Security Operations Center)
  \item Supports detection, alerting, forensic analysis
  \item Dashboards, queries, incident timelines
\end{itemize}
}

\subsection{SIEM Components}
{
\begin{itemize}[noitemsep]
  \item \textbf{Sensors:} Sources that generate security-relevant data for the SIEM
    \begin{itemize}[noitemsep]
      \item \textbf{NIDS (Network Intrusion Detection System):} Monitors network traffic for anomalies (e.g., Snort, Suricata)
      \item \textbf{HIDS (Host Intrusion Detection System):} Monitors system-level activity like file access, login attempts (e.g., OSSEC)
    \end{itemize}

  \item \textbf{Log Collection \& Normalization:}
    \begin{itemize}[noitemsep]
      \item Collect logs from various sources (firewalls, servers, applications)
      \item Normalize into a common structured format (fields: timestamp, source IP, event type, etc.)
      \item Enables correlation and efficient querying
    \end{itemize}

  \item \textbf{Asset Inventory:}
    \begin{itemize}[noitemsep]
      \item List of known systems, owners, IPs, roles, and criticality
      \item Provides essential context for alerts and triage
      \item Supports prioritization of incidents and reduces false positives
    \end{itemize}

  \item \textbf{Vulnerability Scanner:}
    \begin{itemize}[noitemsep]
      \item Scans systems for known weaknesses (CVEs – Common Vulnerabilities and Exposures)
      \item Tools: Nessus, OpenVAS
      \item Results feed into SIEM to help prioritize alerts
    \end{itemize}

  \item \textbf{Correlation Engine:}
    \begin{itemize}[noitemsep]
      \item Central logic unit that links related events to detect complex attacks
      \item \textit{Simple rule:} 5 failed logins $\rightarrow$ brute force detection
      \item \textit{Complex rule:} new login location + privilege change + file access = suspicious behavior
      \item Enables detection of attacker TTPs (Tactics, Techniques, Procedures)
    \end{itemize}
\end{itemize}
}


\subsection{Pyramid of Pain}
{
\begin{itemize}[noitemsep]
  \item Defense model: higher levels = harder for attacker to adapt
  \item Indicators (low to high): Hashes, IPs, Domains, TTPs
  \item Goal: detect \& disrupt attacker TTPs, not just IOCs
\end{itemize}
}

\subsection{SIEM Lab Summary}
{
\textbf{Will not be tested in the exam.}
}
