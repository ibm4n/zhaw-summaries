\pagestyle{empty} % Keine Seitennummern

% Verwendete Pakete

\usepackage[utf8]{inputenc}
\usepackage[top=0.7cm, bottom=0.9cm, left=0.65 cm, right=0.65 cm, ]{geometry}
\usepackage{amsmath}
\usepackage{amsfonts}
\usepackage{lmodern}
\usepackage{enumitem}
\usepackage{graphicx}
\setlength{\parindent}{0pt}
\usepackage[normalem]{ulem}
\usepackage[dvipsnames]{xcolor}
\usepackage{mathabx}
\usepackage{colortbl}
\usepackage[ngerman]{babel}
\usepackage{mathtools}
\usepackage{wallpaper}
\usepackage{changepage}
\usepackage{tikz}
\usepackage{tabularx}
\usepackage{makecell}
\usepackage{tcolorbox}
\usepackage{lipsum}
\usepackage{letltxmacro}
\usepackage{multicol}
\usepackage{calc}
\usepackage{ifthen}
\usepackage{hyperref}
\usepackage{graphicx}
\usepackage{titlesec}
\usepackage{tikz}
\usetikzlibrary{decorations.pathmorphing}
\graphicspath{ {./img/} }
\usepackage{wrapfig}

% paragraph settings
\titlespacing*{\paragraph}{2pt}{0pt}{0.5em}


\newcommand{\hr}{\centerline{\rule{3.5in}{1pt}}}
%\colorbox[HTML]{e4e4e4}{\makebox[\textwidth-2\fboxsep][l]
% Spalteneinstellungen

\setlength\columnsep{3mm}
\setlength{\columnseprule}{0pt}

% Neue Befehle

% Bullet-Symbol für Aufzählungen
\renewcommand\textbullet{\ensuremath{\bullet}}


\def\unsignedbytecalc#1{%
	\par\smallskip
	\noindent$#1_{10}$\par
	\smallskip
	\gdef\result{}%
	$\left.\begin{array}{r@{\quad}|c}\udbc{#1}\end{array}\right\}\result$\par}

\makeatletter

\def\udbc#1{%
\ifnum#1=\z@
	\expandafter\@gobble
\else
	\expandafter\@firstofone
\fi
{\!\underline{\,#1}&\edef\r{\ifodd#1 1\else 0\fi}\r\xdef\result{\r\result}\\
\expandafter\udbc\expandafter{\the\numexpr(\ifodd#1 #1-1\else#1\fi)/2\relax}%
}}



\tikzstyle{mybox} = [draw=black, fill=white, very thick,
rectangle, rounded corners, inner sep=10pt, inner ysep=10pt]
\tikzstyle{fancytitle} =[fill=black, text=white, font=\bfseries]

\tikzstyle{examplebox} = [draw=gray, fill=white, thick,
rectangle, rounded corners, inner sep=4pt, inner ysep=12pt]
\tikzstyle{exampletitle} =[fill=gray, text=black, font=\bfseries]

% Eingekreiste Nummern für Aufzählungen
\newcommand*\circled[1]{\tikz[baseline=(char.base)]{
		\node[shape=circle,draw,inner sep=1.2pt] (char) {#1};}}

% Horizontale Punkte
\LetLtxMacro\orgddots\ddots
\makeatletter
\DeclareRobustCommand\vdots{%
	\mathpalette\@vdots{}%
}
\newcommand*{\@vdots}[2]{%
	% #1: math style
	% #2: unused
	\sbox0{$#1\cdotp\cdotp\cdotp\m@th$}%
	\sbox2{$#1.\m@th$}%
	\vbox{%
		\dimen@=\wd0 %
		\advance\dimen@ -3\ht2 %
		\kern.5\dimen@
		% remove side bearings
		\dimen@=\wd2 %
		\advance\dimen@ -\ht2 %
		\dimen2=\wd0 %
		\advance\dimen2 -\dimen@
		\vbox to \dimen2{%
			\offinterlineskip
			\copy2 \vfill\copy2 \vfill\copy2 %
		}%
	}%
}
\DeclareRobustCommand\ddots{%
	\mathinner{%
		\mathpalette\@ddots{}%
		\mkern\thinmuskip
	}%
}

% Vertikale Punkte
\DeclareRobustCommand\ddots{%
	\mathinner{%
		\mathpalette\@ddots{}%
		\mkern\thinmuskip
	}%
}
\newcommand*{\@ddots}[2]{%
	% #1: math style
	% #2: unused
	\sbox0{$#1\cdotp\cdotp\cdotp\m@th$}%
	\sbox2{$#1.\m@th$}%
	\vbox{%
		\dimen@=\wd0 %
		\advance\dimen@ -3\ht2 %
		\kern.5\dimen@
		% remove side bearings
		\dimen@=\wd2 %
		\advance\dimen@ -\ht2 %
		\dimen2=\wd0 %
		\advance\dimen2 -\dimen@
		\vbox to \dimen2{%
			\offinterlineskip
			\hbox{$#1\mathpunct{.}\m@th$}%
			\vfill
			\hbox{$#1\mathpunct{\kern\wd2}\mathpunct{.}\m@th$}%
			\vfill
			\hbox{$#1\mathpunct{\kern\wd2}\mathpunct{\kern\wd2}\mathpunct{.}\m@th$}%
		}%
	}%
}
\makeatother

% Schriftart
\renewcommand{\familydefault}{\sfdefault}

% Dokument-Info Block	
\newcommand{\DocumentInfo}[3]{
	\begin{tcolorbox}[
			arc=0mm,
			colback = white!38!black,
			boxrule=0pt,
			toptitle=1mm,
			bottomtitle=1mm,
			right=2mm,
			left=2mm,
			leftright skip = -0.5mm,
			title= \huge \center \textbf{#1} \par \large \vskip1mm #2 \par \vskip1mm \small 	Version: \today,
			fontupper=\color{white},
			after skip = 0 mm,
			top=0.1mm,
			bottom=1mm]

		\small #3
		\vskip1mm
	\end{tcolorbox}
}

% Überschrift
\renewcommand{\section}[1]{
	\begin{tcolorbox}[
			arc=0mm,
			colback=white!38!black,
			colframe=white,
			bottomrule = 0 mm,
			toprule = 0 mm,
			leftrule = 0 mm,
			rightrule = 0 mm,
			valign=center,
			left=0.5mm,
			top= 0.7 mm,
			bottom= 0.7 mm,
			fontupper=\color{white},
			before skip = 0mm,
			leftright skip = -0.5mm,
			after skip = 0 mm]

		\textbf{#1}
	\end{tcolorbox}
}

% Abschnitt	
\renewcommand{\subsection}[2]{
	\begin{tcolorbox}[
			arc=0mm,
			colback=white!75!black,
			colframe=white,
			bottomrule = 0 mm,
			toprule = 0 mm,
			leftrule = 0 mm,
			rightrule = 0 mm,
			valign=center,
			left=0.5mm,
			top=0.2mm,
			bottom=0.2mm,
			before skip = 0mm,
			leftright skip = -0.5mm,
			after skip = 1.4 mm]

		\small \textbf{#1}
	\end{tcolorbox}

	\begin{adjustwidth}{0.5mm}{1mm}
		\small
		#2
		\vspace{0.5mm}
	\end{adjustwidth}
}

% Weisser Balken zwischen Abschnitten
\newcommand{\WhiteSpace}[0]{
	\begin{tcolorbox}[
			arc=0mm,
			colback=white,
			colframe=white,
			bottomrule = 0 mm,
			toprule = 0 mm,
			leftrule = 0 mm,
			rightrule = 0 mm,
			valign=center,
			left=0.5mm,
			top= -0.2 mm,
			bottom= -0.2 mm,
			fontupper=\color{white},
			before skip = 0mm,
			leftright skip = -0.5mm,
			after skip = 0 mm]

	\end{tcolorbox}
}

% Hintergrundbild (graue Spalten)

%	\CenterWallPaper{1}{0_Setup/background.pdf}

% TabularX Zeug (Paket für Tabellen)

\newcolumntype{C}[1]{>{\centering\arraybackslash}p{#1}}

% TikZ Zeug (Paket für Vektorgraphiken)

\usetikzlibrary{decorations.pathreplacing,calc}

\newcommand{\tikzmark}[2][-3pt]{\tikz[remember picture, overlay, baseline=-0.5ex]\node[#1](#2){};}

\tikzset{brace/.style={decorate, decoration={brace}},
	brace mirrored/.style={decorate, decoration={brace,mirror}},
}

\newcounter{brace}
\setcounter{brace}{0}
\newcommand{\drawbrace}[3][brace]{%
	\refstepcounter{brace}
	\tikz[remember picture, overlay]\draw[#1] (#2.center)--(#3.center)node[pos=0.5, name=brace-\thebrace]{};
}

\newcommand{\annote}[3][]{%
	\tikz[remember picture, overlay]\node[#1] at (#2) {#3};
}