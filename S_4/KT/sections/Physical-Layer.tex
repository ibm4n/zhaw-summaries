\newpage
\section{Physical Layer}

\subsection{Arten der Kommunikation (Verkehrsbeziehung)}{
    \begin{itemize}[noitemsep]
        \item Simplex $\to$ Ein Kanal, in eine Richtung
        \item Halbduplex $\to$ Ein Kanal, abwechslungsweise in zwei Richtungen
        \item Vollduplex $\to$ Ein Kanal pro Richtung
    \end{itemize}
}


\subsection{Arten der Verbindungen (Kopplung)}

\paragraph{Punkt - Punkt}{

    {Direkte Verbindung zweier Kommunikationspartner }

    \includegraphics[scale=.35]{img/kopplung.png}
}

\paragraph{Shared Medium }{
    {Mehrere Partner verwenden das gleiche Medium   \\}

    \includegraphics[scale=.35]{img/kopplung_2.png}

}

\columnbreak
\subsection{Serielle asynchrone Übertragung}{

    \includegraphics[scale=.25]{img/async.png}
    {$LSB$ = Least Significant Bit, $MSB$ = Most Significant Bit}\\

    \begin{tikzpicture}
        \node [examplebox] (box){
            \begin{minipage}{0.3\textwidth}
                Übertragener Wert ablesen: \\
                $LSB$ zuerst, $MSB$ zuletzt \\
                $  1101`0100 \to LSB $ zuerst $ \to 0100`1101 $
            \end{minipage}
        };
        \node[exampletitle, right=8pt] at (box.north west) {Wichtig:};
    \end{tikzpicture}


}

\subsection{Serielle synchrone Übertragung}{
    \includegraphics[scale=.275]{img/sync.png}
}


\subsection{Datenübertragungsrate}{
    \begin{itemize}[noitemsep]
        \item Baudrate $\to$ Symbole pro Sekunde
        \item Zeichenrate $\to$ Zeichen pro Sekunde
    \end{itemize}
}

\subsection{Frequenz}{
    {Die Frequenz ist die Anzahl der Schwingungen pro Sekunde.\\
            Masseinheit Hertz (Hz)\\}
}

\subsection{Bit-Dauer }
{  T [s] = Bit-Dauer, B = Baud \\}
$$ T = \frac{1}{B}$$

\columnbreak
\subsection{maximale Symbolrate}
{    Die maximale Symbolrate $f_s$ (Baud) ist gleich der doppelten Bandbreite B (Hz) des
    Übertragungskanals.
}
{Einheit: Baud (Bd)}

{Nyquist:}
$$ f_s = 2 \cdot B$$

\subsection{Maximal erreichbare Bitrate}{
R [bit/s] = Bitrate \\
$$ R \leq 2B \cdot log_2{M} $$
$$ log_2(x) = \frac{log_{10}(x)}{log_{10}(2)} $$
}



\subsection{Bandbreite}{
    Die Bandbreite hängt von der Übertragungsstrecke und der Stärke des Signals im
    Vergleich zu den vorhandenen Störungen, ab.
    \begin{itemize}[noitemsep]
        \item Eigenschaft des Übertragungskanals und durch das Medium begrenzt
        \item Masseinheit Hertz (Hz)
    \end{itemize}
}

\subsection{Kanalkapazität}{
    Berücksichtigt für einen realen Kanal das Signal-zu-Rausch Leistungverhältnis S/N (Shannon)
    \\ Einheit Bit/s (bps)
    $$ C_s = B \cdot log_2(1 + \frac{S}{N})$$
    $$ log_2(x) = \frac{log_{10}(x)}{log_{10}(2)} $$
}
